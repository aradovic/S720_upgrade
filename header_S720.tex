

\usepackage{lscape}
\newcommand{\blandscape}{\begin{landscape}}
\newcommand{\elandscape}{\end{landscape}}

\usepackage{geometry}
\geometry{a4paper,footskip=1 cm,twoside} %kako bi funkcioniralo odd i even što se tiče strana i što se tiče LE, RO

\usepackage[width=.9\textwidth]{caption} %ovo je za podesiti širinu tablice u odnosu na tekst
%\usepackage{floatrow} da ga natjeramo da stavi caption dole
%\floatsetup[table]{capposition=bottom}  %da se natjera caption da bude dole jer to mu inače nije prirodno (ne treba s xtable opcijom ali xtable ne radi u wordu)

\usepackage[dvipsnames,table]{xcolor} %da možee raditi background color i font color u pixiedust paketu u R-u

\usepackage{longtable}

\usepackage{rotating}
\usepackage{multicol}

\usepackage{tablestyles}
\disablealternatecolors


\usepackage{hyperref} %ovdje nije bilo potrebno ali ovo je ako slučajno pokupi druge boje linkova
\hypersetup{colorlinks = false}

\usepackage{layout}
\usepackage{array}

\usepackage{titling}

\usepackage{lmodern} %dodatni fontovi

\usepackage{graphicx} 

\usepackage{fancyhdr}

\usepackage{wrapfig}

\usepackage{placeins}

\usepackage{sectsty} %paket pomoću kojeg definiramo boje određenih razina teksta
\definecolor{srcecrvena}{RGB}{192,0,0}
\chapterfont{\color{srcecrvena}} 
\sectionfont{\color{srcecrvena}}
\subsectionfont{\color{srcecrvena}}
\subsubsectionfont{\color{srcecrvena}}

\usepackage{tabularx}

\usepackage{multicol}%za staviti onu crtu kod CC

\usepackage[titletoc]{appendix} %paket da nam radi drugačije numeriranje appendix-a

\setlength{\columnseprule}{1.5pt}
\def\columnseprulecolor{\color{black}}

\addtocontents{toc}{\protect\thispagestyle{empty}}

\usepackage{titling}



\pagestyle{fancy}

\fancyhf{} %ovim ubijemo sve postavke fancy pagestyle-a da bi mogli staviti nove
%\fancyhead[L]{\nouppercase{Upoznavanje sa sintaksom jezika R i njegova primjena u osnovnoj statističkoj i grafičkoj analizi podataka (S72000)}}


\usepackage{tabularx}


\lhead{Upoznavanje sa sintaksom jezika R i njegova primjena u osnovnoj statističkoj i grafičkoj analizi podataka (S200)}

\rhead{Upoznavanje sa sintaksom jezika R i njegova primjena u osnovnoj statističkoj i grafičkoj analizi podataka (S200)}



%\cfoot{\includegraphics[width = .05\textwidth]{srce_400x400}}

%\rfoot{\thepage}


\renewcommand{\headrulewidth}{1pt}
\renewcommand{\footrulewidth}{0.1pt}

\fancyhf{}
\fancyhead[RO]{\nouppercase{\rightmark}}
\renewcommand{\rightmark}{\normalsize Upoznavanje sa sintaksom jezika R i njegova primjena u osnovnoj statističkoj i grafičkoj analizi podataka (S200) }
%\,\textbar\,
%\includegraphics[width = .05\textwidth]{srce_header}}
\fancyhead[LE]{\nouppercase{\rightmark}}
\renewcommand{\leftmark}{\normalsize Upoznavanje sa sintaksom jezika R i njegova primjena u osnovnoj statističkoj i grafičkoj analizi podataka (S200) }


\cfoot{\includegraphics[width = .05\textwidth]{srce_400x400}}
\fancyfoot[R]{\thepage}




\renewcommand{\figurename}{Slika }
\renewcommand{\contentsname}{Sadržaj}
\renewcommand{\tablename}{Tablica }








	


